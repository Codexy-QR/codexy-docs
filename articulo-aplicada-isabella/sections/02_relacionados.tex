\section{Marco teórico y trabajos relacionados}

La literatura sobre códigos QR y gestión de inventarios ha evolucionado desde la invención de los QR en 1994. Soon (2008) establece el marco teórico de los QR como estándar internacional, destacando su robustez y capacidad de almacenamiento. Este trabajo sienta las bases para aplicaciones prácticas en diversos campos.

En gestión de inventarios, Mandala y Susanto (2023) demuestran la efectividad de QR en entornos gubernamentales, logrando 100\% de precisión en pruebas. Paty Quispe (2020) integra QR con metodologías ágiles como XP, aplicadas a fundaciones. Maharjan (2018) combina QR con bases de datos para automatización de RRHH e inventarios.

En sectores específicos, Sánchez-Gómez et al. (2021) aplican QR en retail, reduciendo pérdidas. Amaya et al. (2018) vinculan QR con modelos económicos como EOQ. En salud, Carrillo-Larco y Curioso (2013) y Méndez Olivero et al. (2024) usan QR para diseminación y control de medicamentos.

En agricultura, Silva Peñafiel et al. (2022) y Vargas Calderón y González Barrera (2016) implementan QR para trazabilidad. En educación, Graván y Gutiérrez (2014) y Pérez (2023) potencian aprendizaje con QR.

Otros trabajos incluyen Dey (2012) en criptografía con QR, y Paz Enrique (2017) en instituciones de información.

Se identifican vacíos en síntesis integrales que combinen estos estudios en una guía unificada, lo cual aborda este artículo mediante análisis cuantitativo y reflexiones integradas.