\section{Introducción}

En la era digital, la gestión de inventarios representa un desafío crítico para organizaciones de diversos sectores, desde la industria manufacturera hasta la salud y la agricultura. Los métodos tradicionales basados en papel y Excel generan ineficiencias, pérdidas económicas y errores humanos, como se evidencia en múltiples estudios. La adopción de tecnologías innovadoras, como los códigos QR, ofrece una solución accesible y efectiva para modernizar estos procesos.

Este artículo sintetiza 20 investigaciones aplicadas que exploran la implementación de códigos QR en sistemas de gestión de inventarios. Desde su invención en 1994 por Denso, los QR han evolucionado como estándar internacional (ISO/IEC 18004), permitiendo almacenar información compleja de forma rápida y precisa. Los proyectos analizados demuestran su aplicación en etiquetado de activos, trazabilidad de productos, control de stock y optimización de cadenas de suministro.

El problema central abordado es la transición de procesos manuales a digitales, reduciendo costos operativos y mejorando la precisión. Por ejemplo, en agencias gubernamentales y farmacias, los QR han logrado reducciones significativas en pérdidas y tiempos de auditoría. En agricultura, facilitan el monitoreo de cultivos, y en educación, potencian el aprendizaje interactivo.

La motivación de este trabajo radica en la necesidad de una revisión sistemática que integre experiencias prácticas de 20 artículos, proporcionando una guía unificada para la implementación de QR en inventarios. Siguiendo principios de DevOps y métricas objetivas, evaluamos el impacto en eficiencia, sostenibilidad y escalabilidad.

Nuestras contribuciones incluyen: (1) una síntesis comprehensiva de tecnologías QR aplicadas a inventarios, (2) análisis cuantitativo con porcentajes de efectividad derivados de los estudios, y (3) reflexiones integradas que derivan en recomendaciones prácticas para equipos de desarrollo e implementación.

El artículo está estructurado como sigue: tras esta introducción, se revisa el trabajo relacionado; se describe la metodología de síntesis; se detalla la implementación; se presentan resultados con métricas; se discute implicaciones; y se concluye con lecciones aprendidas.