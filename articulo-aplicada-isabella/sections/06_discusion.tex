\section{Discusión}

Interpretamos los efectos de los códigos QR en la gestión de inventarios, su generalización a otros sectores y los costos/beneficios. Comparamos con literatura existente y discutimos implicaciones para equipos de desarrollo, organizaciones y decisores.

\subsection{Efectos observados}

Los resultados muestran que los QR reducen errores humanos, mejoran la trazabilidad y optimizan procesos. En sectores como salud y agricultura, previenen pérdidas y aseguran cumplimiento normativo. Sin embargo, la efectividad depende del contexto; en entornos con baja adopción tecnológica, los beneficios son menores.

\subsection{Generalización y limitaciones}

Los hallazgos se generalizan a industrias manufactureras y retail, pero limitaciones incluyen dependencia de dispositivos móviles y necesidad de capacitación. Comparado con RFID, QR es más económico pero menos resistente a daños.

\subsection{Costos y beneficios}

Beneficios: reducción de costos operativos (hasta 70\%), aumento de eficiencia (95\% promedio). Costos: implementación inicial, mantenimiento de bases de datos. El ROI es positivo en la mayoría de casos.

\subsection{Implicaciones}

Para equipos: adoptar metodologías ágiles para integración QR. Para organizaciones: invertir en digitalización. Para decisores: priorizar tecnologías accesibles como QR para transformación digital.