Este artículo sintetiza 20 artículos aplicados sobre la implementación de códigos QR en sistemas de gestión de inventarios, destacando su eficacia en diversos sectores como salud, agricultura, educación e industria. Se analiza la arquitectura de sistemas integrados con QR, tecnologías empleadas (aplicaciones móviles, bases de datos, metodologías ágiles), y resultados cuantitativos que muestran porcentajes de efectividad superiores al 90\% en la mayoría de casos. La metodología combina revisión sistemática de literatura con síntesis de experiencias prácticas, evaluando beneficios como reducción de pérdidas, mejora en trazabilidad y optimización de procesos. Se discuten implicaciones para DevOps, sostenibilidad y futuras tendencias, concluyendo que los códigos QR representan una solución accesible y transformadora para la digitalización de inventarios. Palabras clave: códigos QR, gestión de inventarios, sistemas de información, DevOps, trazabilidad.