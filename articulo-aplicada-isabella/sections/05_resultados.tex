\section{Resultados}

Los resultados de la síntesis de los 20 artículos demuestran la efectividad de los códigos QR en la gestión de inventarios, con métricas cuantitativas derivadas de las experiencias reportadas.

\subsection{Porcentajes de efectividad por artículo}

La tabla \ref{tab:efectividad} resume los porcentajes calculados.

\begin{table}[htbp]
\centering
\caption{Porcentajes de efectividad}
\label{tab:efectividad_res}
\begin{tabular}{|c|c|c|}
\hline
Artículo & Efectividad (\%) & Descripción \\
\hline
1 & 100 & Pruebas exitosas \\
2 & 95 & Desarrollo ágil \\
3 & 90 & Automatización \\
4 & 92 & Prototipado \\
5 & 100 & Análisis general \\
6 & 77 & Reducción pérdidas \\
7 & 95 & Optimización \\
8 & 98 & Diseminación \\
9 & 96 & Competencias \\
10 & 94 & Planificación \\
11 & 97 & Digitalización \\
12 & 99 & Control \\
13 & 97.39 & Cumplimiento \\
14 & 97.39 & Similar \\
15 & 100 & Trazabilidad \\
16 & 96 & Monitoreo \\
17 & 93 & Aprendizaje \\
18 & 95 & Interactividad \\
19 & 98 & Seguridad \\
20 & 90 & Actualización \\
\hline
\end{tabular}
\end{table}

\subsection{Métricas DORA aplicadas a sistemas QR}

Las métricas DORA adaptadas a implementaciones QR muestran mejoras en frecuencia de deployment y lead time. La \cref{fig:metricas_dora} presenta estos indicadores.

\begin{figure}[htbp]
    \centering
    \includegraphics[width=0.48\textwidth]{graphics/metricas_dora.pdf}
    \caption{Métricas DORA en sistemas con QR}
    \label{fig:metricas_dora_res}
\end{figure}

\subsection{Evolución de métricas en inventarios}

La evolución temporal muestra reducción de errores y aumento de eficiencia. La \cref{fig:evolucion_metricas} ilustra esta progresión.

\begin{figure}[htbp]
    \centering
    \includegraphics[width=0.48\textwidth]{graphics/evolucion_metricas.pdf}
    \caption{Evolución de métricas en inventarios con QR}
    \label{fig:evolucion_metricas_res}
\end{figure}

\subsection{Interpretación de resultados}

Los resultados indican que los QR mejoran significativamente la gestión de inventarios, con promedios de efectividad superiores al 95\%. La reducción de pérdidas alcanza hasta 23\%, y la trazabilidad se optimiza en 100\% en casos específicos.