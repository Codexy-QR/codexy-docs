\appendix
\section{Resúmenes completos de los 20 artículos}

Aquí se incluyen los resúmenes completos de los 20 artículos analizados, para referencia detallada.

\subsection{1. Pengembangan Sistem Inventaris Barang Berbasis QR Code pada Badan Kepegawaian Daerah Provinsi Bengkulu}
\textbf{Resumen:} La Agencia de Personal en Bengkulu tenía un lío tremendo con su inventario manual en Excel, causando pérdida de datos y monitoreo casi imposible. Para resolverlo, se propuso un sistema web con Códigos QR. La idea fue etiquetar cada activo con un QR que guarda su información, permitiendo que cualquiera lo escanee con un smartphone para ver los datos al instante. Tras probarlo (Black Box Testing), el resultado fue 100\% de efectividad, lo que demuestra que este cambio digital es la solución definitiva para un control de activos rápido y preciso.

\textbf{Reflexión:} Lo que aprendemos de esto es que invertir en tecnología simple, como una etiqueta QR, es el mejor atajo para deshacerse de procesos viejos como el Excel. Si el sistema alcanza esa efectividad, queda claro que mantener la gestión manual.

\textbf{Referencia:} Mandala, R. C., \& Susanto, A. (2023). Pengembangan Sistem Inventaris Barang Berbasis QR Code pada Badan Kepegawaian Daerah Provinsi Bengkulu. Jurnal Pustaka AI (Pusat Akses Kajian Teknologi Artificial Intelligence), 3(1), 47–51.

\subsection{2. APLICACIÓN MÓVIL DE CONTROL DE INVENTARIOS MEDIANTE CÓDIGO QR}
\textbf{Resumen:} El proyecto se centró en cómo modernizar el control de inventarios de la Fundación Ladislao Calani (CDI BO0450), que en pleno siglo XXI seguía lidiando con procesos manuales. El problema era obvio: los activos se perdían, había demoras en las auditorías y nunca se sabía quién había pedido algo ni dónde estaba. La solución que propuso este Proyecto de Grado fue crear una aplicación móvil Android que hiciera el trabajo pesado. Utilizando la metodología Extreme Programming (XP) para un desarrollo ágil, se diseñó una aplicación que maneja el registro y el préstamo de activos simplemente escaneando un Código QR pegado a cada artículo. La aplicación funciona con conexión a internet para interactuar con la base de datos (PHP y MySQL).

\textbf{Reflexión:} Lo que veo aquí es un claro ejemplo de que la tecnología sencilla puede ser más efectiva que lo complicado. Un simple Código QR, manejado desde un celular, resuelve un dolor de cabeza gigante de pérdidas económicas y burocracia.

\textbf{Referencia:} Paty Quispe, R. O. G. E. R. D. (2020). APLICACIÓN MÓVIL DE CONTROL DE INVENTARIOS MEDIANTE CÓDIGO QR. [Tesis de Licenciatura].

\subsection{3. AN INVENTORY AND HUMAN RESOURCE MANAGEMENT SYSTEM}
\textbf{Resumen:} Lo que buscaban con este proyecto era para ayudar a llevar el inventario y dejar de controlar a los empleados a la antigua, o sea, con puro papel y Excel. Entonces, se creo un sistema de escritorio (IMS). La genialidad es que ataca dos frentes: en Recursos Humanos, ya nadie tiene que pasarse horas calculando las planillas; el software registra la entrada y salida, y te da las horas mensuales automáticamente. El administrador puede ver, medir o sacar stock de la tienda de una forma súper rápida. Lo hicieron con Java y MySQL, lo probaron en Windows y funcionó bien.

\textbf{Reflexión:} Este proyecto prueba que una solución tecnológica sencilla (incluso si es una aplicación de escritorio).

\textbf{Referencia:} Maharjan, M. (2018). AN INVENTORY AND HUMAN RESOURCE MANAGEMENT SYSTEM. [Tesis de Licenciatura].

\subsection{4. Sistem Pengurusan Inventori (Online Bundle Store Inventory Management System)}
\textbf{Resumen:} El problema principal de la tienda Kedai Bundle Amin, que vende calzado y ropa de segunda mano, era el lío que tenían con su inventario. Como lo manejaban todo a mano, perdían tiempo, el stock no estaba seguro y no podía identificar rápidamente qué productos se estaban agotando. Por eso, propusieron desarrollar un Sistema de Gestión de Inventario en Línea, usando la metodología prototipado para asegurar que el sistema realmente fuera útil. La idea era simple: reemplazar ese proceso manual por un sistema web que permitiría a los trabajadores actualizar la información del stock y ayudara al gerente a identificar el inventario bajo mínimos.

\textbf{Reflexión:} Los métodos manuales (como el papel) ya no es tolerable en ningún negocio, por muy específico que sea.

\textbf{Referencia:} Shamsuddin, M. A. M., \& Salamat, M. A. (2021). Sistem Pengurusan Inventori dalam Talian Kedai Bundle. Applied Information Technology And Computer Science, 2(2), 1799–1809.

\subsection{5. QR Code (Análisis del Código QR)}
\textbf{Resumen:} El Código QR, inventado por Denso en 1994, representa la evolución más importante en la identificación de datos, y por eso es un estándar internacional (ISO/IEC 18004). Nació porque los códigos de barras comunes eran ineficientes para guardar la gran cantidad de información que se necesitaba. Su genialidad reside en ser súper robusto y rápido: se puede leer desde cualquier ángulo (360°), es resistente al daño y a las manchas y lo más importante, su patente es libre, lo que significa que cualquiera puede usarlo sin pagar un solo peso. Gracias a estas ventajas, se ha convertido en una solución esencial para todo, desde la gestión de pacientes en hospitales y el rastreo de ganado.

\textbf{Reflexión:} Con el Código QR fue un golpe de genialidad total, no solo un invento geek. Nos han demostrado que la tecnología más top es esa que te resuelve un problema complejo con la mayor sencillez del mundo.

\textbf{Referencia:} Soon, T. J. (2008). QR code. Synthesis Journal, 2008(3), 59–78.

\subsection{6. Sistema de control de inventarios aplicando códigos QR}
\textbf{Resumen:} La zapatería Odalys Zapatería y Accesorios era la misma de siempre: perdían una cantidad brutal de mercancía (hasta un 23\% en 2020) porque el control de inventario era un desastre, y eso se complicaba con la rotación de personal. La solución que plantearon fue dejar de usar los códigos de barras de siempre e implementar un sistema de control de inventarios basado en Códigos QR para poner orden en el almacén. Primero, estandarizaron todo con órdenes de compra para ver qué entraba, y luego generaron los códigos QR en masa (usando una herramienta online) para los productos.

\textbf{Reflexión:} La ineficiencia generada por los métodos manuales ya no es tolerable. Estos proyectos prueban que una solución tecnológica sencilla.

\textbf{Referencia:} Sánchez-Gómez, W. A., Zapata-Rebolloso, A. Z., \& Gutiérrez-Zozaya, S. J. (2021). Sistema de control de inventarios aplicando códigos QR. Revista Interdisciplinaria de Ingeniería Sustentable y Desarrollo Social, 7(1), 287–296.

\subsection{7. IMPLEMENTACION DE CODIGOS QR EN UN SISTEMA GESTOR DE INVENTARIOS (EOQ)}
\textbf{Resumen:} El objetivo central de este proyecto era vincular el modelo de inventario clásico, la Cantidad Económica de Pedido (EOQ) (que ayuda a saber con exactitud cuánto y cuándo pedir) con la tecnología moderna del Código QR para revolucionar el control de inventarios. La idea es simple: se diseñó un software que usa códigos QR para codificar la información de los productos y luego la carga a una base de datos. De esta forma, el sistema te permite llevar conteos, checar el stock y saber el pedido óptimo (gracias al EOQ) de forma rápida.

\textbf{Reflexión:} Lo que destaca de este enfoque es la inteligencia de mezclar un modelo económico clásico, como el EOQ.

\textbf{Referencia:} Amaya, L. F., Tiboche, F. J., Ruiz, E. T., \& Carreño, D. A. (2018). Implementación de códigos QR en un sistema gestor de inventarios. [Artículo de Congreso].

\subsection{8. OPORTUNIDADES DEL CÓDIGO QR PARA DISEMINAR INFORMACIÓN EN SALUD}
\textbf{Resumen:} Es una Carta al Editor que busca resaltar las oportunidades del Código QR para la diseminación de información en el sector salud, dado que su uso es aún incipiente en este campo en Perú. El Código QR (Quick Response) es superior a los códigos de barras tradicionales porque almacena mayor información y permite acceder a diversos recursos digitales (webs, multimedia, mapas) mediante el escaneo con un teléfono móvil. El Ministerio de Salud del Perú (MINSA) ha sido pionero en usarlo en publicaciones como el "Compendio de estadísticas de hechos vitales" para facilitar el acceso rápido a la información.

\textbf{Reflexión:} El QR es una herramienta increíblemente barata y súper práctica para darle información directa a la gente.

\textbf{Referencia:} Carrillo-Larco, R. M., \& Curioso, W. H. (2013). Oportunidades del código QR para diseminar información en salud. Revista Peruana de Medicina Experimental y Salud Pública, 30(2), 362–363.

\subsection{9. Social networks as tools for acquiring competences at university: QR codes through Facebook}
\textbf{Resumen:} Este estudio que juntó estudiantes de España y Venezuela, era ver cómo las redes sociales ayudaban a los universitarios a desarrollar habilidades que no se enseñan en el currículo formal. Lo que hicieron fue crear un repositorio virtual en Facebook para que una muestra de 175 estudiantes compartiera y aprendiera sobre Códigos QR y sus usos. Las conclusiones fueron súper claras: los alumnos vieron Facebook como una herramienta cercana, fácil de usar y muy fiable para estudiar y compartir información de forma colaborativa, superando las ventajas de otros entornos. Además, se comprobó que esta dinámica (usar Facebook con Códigos QR) es genial para que desarrollen y refuercen tres tipos de competencias clave: las instrumentales (como el análisis y la síntesis), las interpersonales (como el trabajo en equipo internacional) y las sistémicas (como la motivación y la creatividad para investigar).

\textbf{Reflexión:} La universidad necesita dejar lo cuadrado. Al mezclar el QR con plataformas que los estudiantes sí usan, el aprendizaje se vuelve real y útil, desarrollando habilidades blandas vitales.

\textbf{Referencia:} Graván, P. R., \& Gutiérrez, Á. M. (2014). Las redes sociales como herramientas para la adquisición de competencias en la universidad: los códigos QR a través de Facebook. RUSC. Revista Universidades y Sociedad del Conocimiento, 11(2), 27–42.

\subsection{10. A Simple Case Study of Material Requirement Planning (MRP)}
\textbf{Resumen:} Es la herramienta que usan las fábricas para no tener que adivinar. El MRP es un sistema que toma la lista de materiales, el stock actual y el calendario de producción, y te dice exactamente qué, cuánto y cuándo tienes que pedir cada pieza. La meta es balancear dos cosas: tener un nivel de servicio óptimo (que nunca falte nada para producir) y, a la vez, reducir al máximo los costos y el dinero inmovilizado en el inventario. Este artículo, en particular, se propuso resolver un problema práctico de ensamblaje (usando como ejemplo un coche) para demostrar, con números y cálculos detallados, cómo funciona el MRP. La conclusión es que, al practicar el MRP, la empresa puede controlar el inventario de forma fácil, ahorrar mucho dinero, aprovechar descuentos por volumen y reducir los problemas logísticos.

\textbf{Reflexión:} Si estás en producción, tienes que usar el MRP. No puedes darte el lujo de "adivinar" el inventario. Este sistema es vital porque te obliga a planificar la producción y la compra a la vez.

\textbf{Referencia:} Sarkar, A., Das, D., Chakraborty, S., \& Biswas, N. (2013). A simple case study of material requirement planning. IOSR Journal of Mechanical and Civil Engineering, 9(5), 58–64.

\subsection{11. PharmaScan: prototipo de aplicación web para la gestión de inventario y facturación de medicamentos con tecnología QR en farmacias de Barranquilla}
\textbf{Resumen:} El punto de partida del proyecto es que la gestión de inventario y la facturación de medicamentos en las farmacias de Barranquilla son un desastre: usan cuadernos manuales, lo que causa errores, demoras y pérdidas. Para cortar con ese problema, propusieron desarrollar PharmaScan, un prototipo de plataforma web que lo automatice todo usando Códigos QR. La plataforma digitaliza el stock, permite escanear productos para un seguimiento detallado y ofrece herramientas para la gestión de caducidades.

\textbf{Reflexión:} La tecnología puede salvar vidas y negocios. Los métodos manuales en farmacias son un riesgo (por los productos vencidos y los errores), así que un sistema con QR es más que un lujo, es una necesidad urgente.

\textbf{Referencia:} Méndez Olivero, A. M., Campo De La Ossa, E., \& De Ávila, L. (2024). PharmaScan: prototipo de aplicación web para la gestión de inventario y facturación de medicamentos con tecnología QR en farmacias de Barranquilla. [Artículo de Congreso].

\subsection{12. Sistema de Control de Personal e Inventarios (U. de La Habana)}
\textbf{Resumen:} El problema en la Universidad de La Habana era que tareas críticas, como el control de acceso de personal o la gestión de inventario, se hacían con listas de papel y métodos anticuados, poniendo en riesgo la seguridad y el orden por culpa del error humano. Para acabar con este desorden, se propuso y desarrolló un sistema digitalizado y sostenible centrado en una API REST (hecha con Python y FastAPI). Esta API actúa como el cerebro modular que centraliza todo: permite al personal de seguridad verificar el acceso escaneando un Código QR que se genera desde el sistema, dándoles una respuesta instantánea. Además, facilita el control de inventario y la creación de un software que puede ampliarse en el futuro.

\textbf{Reflexión:} A mí lo que me queda claro es que la eficiencia no puede ser opcional. Es obvio que la seguridad no puede depender de que un vigilante revise una lista interminable.

\textbf{Referencia:} Solís Fernández, J. A. (2022). Sistema de Control de Personal e Inventarios. [Tesis de Licenciatura]. Universidad de La Habana.

\subsection{13. Implementación de un modelo de gestión de actualización tecnológica en recursos de información a través de código QR para equipo biomédico nivel IIA y IIB, usados en ambulancias básicas, medicalizadas y consultorios de las sanidades aeroportuarias en la empresa Aerosanidad S.A.S.}
\textbf{Resumen:} La gestión de equipos médicos de alto riesgo en ambulancias y sanidades aeroportuarias era caótica debido a la cantidad de manuales y documentos que exigía la ley colombiana. Para resolver esto, se implementó un modelo de gestión que centralizó todo el historial de los equipos (hojas de vida, mantenimientos, videos de uso) en una nube (One Drive). Luego, se generaron Códigos QR que se adhirieron a cada equipo, enlazando directamente a esa carpeta de información. El proyecto fue un éxito: demostró una funcionalidad de lectura del 97.39\%, lo cual garantiza el cumplimiento normativo, agiliza las auditorías y asegura que el personal médico tenga la información vital al alcance de la mano.

\textbf{Reflexión:} Este proyecto es la prueba reina de que la innovación no tiene que ser costosa. En un entorno de alto riesgo como las ambulancias, donde el tiempo es oro y la ley te exige tener toda la información a la mano, un simple Código QR que enlaza a una carpeta en la nube es una solución genial y baratísima.

\textbf{Referencia:} Salazar Meneses, M. P. (2022). Implementación de un modelo de gestión de actualización tecnológica en recursos de información a través de código QR para equipo biomédico nivel IIA y IIB, usados en ambulancias básicas, medicalizadas y consultorios de las sanidades aeroportuarias en la empresa Aerosanidad S.A.S. [Tesis de Grado].

\subsection{14. Implementación de un modelo de gestión de actualización tecnológica en recursos de información a través de código QR para equipo biomédico nivel IIA y IIB, usados en ambulancias básicas, medicalizadas y consultorios de las sanidades aeroportuarias en la empresa Aerosanidad S.A.S. (Duplicado del Art. 13)}
\textbf{Resumen:} La gestión de la información en el sector salud en Colombia es "súper estricta" (por normativas como la Resolución 4725), pero la cantidad de documentos y el historial de mantenimiento de los equipos biomédicos de alto riesgo (en ambulancias y sanidades aeroportuarias) se vuelve imposible de manejar a mano. Para resolver ese rollo, la empresa Aerosanidad S.A.S implementó un modelo de gestión basado en Códigos QR. El proceso fue intensivo: primero, actualizaron y centralizaron todo el historial de los equipos (hojas de vida, órdenes de mantenimiento y hasta videos de capacitación) en una nube (One Drive). Luego, generaron códigos QR que se adhirieron a cada equipo, enlazando directamente a esa carpeta de información. El resultado fue un éxito con un 97.39\% en la lectura, lo que garantiza el cumplimiento normativo en auditorías y asegura que el personal médico tenga la información vital al instante con solo escanear. La conclusión es que esta solución económica y práctica es vital para la seguridad del paciente y la calidad del servicio.

\textbf{Reflexión:} Este proyecto es la prueba de que la innovación no tiene que salir carísima. En un entorno tan crítico como las ambulancias y las sanidades aeroportuarias, donde la ley te exige tener toda la información a la mano.

\textbf{Referencia:} Salazar Meneses, M. P. (2022). Implementación de un modelo de gestión de actualización tecnológica en recursos de información a través de código QR para equipo biomédico nivel IIA y IIB, usados en ambulancias básicas, medicalizadas y consultorios de las sanidades aeroportuarias en la empresa Aerosanidad S.A.S. [Tesis de Grado].

\subsection{15. Diagnóstico de lectura de códigos QR para llevar información de la trazabilidad}
\textbf{Resumen:} La empresa Agrícola Bananera y Exportadora 2 Hermanos no le da a sus clientes ninguna información sobre la trazabilidad del banano que exporta. Esto es un problema, porque los estudios confirman que los consumidores sí creen que es importantísimo saber de dónde viene lo que consumen y cómo se procesa. Por eso, el objetivo de este estudio fue hacer un diagnóstico para ver qué data debía proporcionar la empresa y justificar la implementación de una solución tecnológica. Se encuestó a 16 personas de la empresa y los resultados fueron contundentes: el 100\% dijo que no se proporciona esa información actualmente, y la gran mayoría cree que si se usa una app móvil para escanear Códigos QR y ver la trazabilidad del banano (hacienda, fecha de cosecha, exportadora, etc.), las ventas aumentarían mucho. En resumen, se necesita automatizar con QR la trazabilidad, creando una app web para generar los códigos y otra móvil para que el consumidor los lea.

\textbf{Reflexión:} No es justo que la gente coma algo sin saber de dónde salió, y eso hace que desconfíen. El estudio lo prueba: la empresa está perdiendo plata por no dar la cara.

\textbf{Referencia:} Silva Peñafiel, G. E., Caicedo Villamarin, S. D., Flores Lescano, Á. A., \& Cusco Vinueza, V. A. (2022). Diagnóstico de lectura de códigos QR para llevar información de la trazabilidad. Polo del Conocimiento: Revista científico-profesional, 7(8), 2500–2513.

\subsection{16. Uso del código QR para el seguimiento de la información del eslabón de producción de palma africana en el departamento del Casanare municipio de Villanueva mediante una prueba piloto en la finca Malybu}
\textbf{Resumen:} El problema de la producción de palma africana en el Casanare era que los procesos eran ineficientes, engorrosos y dependían de libros o archivos (que se perdían fácil), lo que generaba retrasos y baja calidad en el fruto. Para resolver esto, el objetivo fue demostrar cómo el Código QR podía corregir y optimizar los eslabones de la cadena productiva. Se hizo una prueba piloto en la finca Malybu, donde se pegaron códigos QR a las plantas para monitorear cada tarea (plateo, abono, polinización, cosecha) y su tiempo de ejecución. Los resultados fueron contundentes: cuando había un seguimiento riguroso con el QR (lote 1), la producción aumentaba; cuando el seguimiento era bajo (lote 3), la producción disminuía por errores en la mano de obra. La conclusión es que el QR brinda un control masivo y actualizado de los procesos, lo que permite detectar y reestructurar fallos sobre la marcha, aumentando la productividad y reduciendo los costos.

\textbf{Reflexión:} Este estudio es clarísimo: la tecnología no es solo para oficinas. Es absurdo que un cultivo tan valioso como la palma africana se maneje con libros de papel, arriesgándose a perder toda la información crítica.

\textbf{Referencia:} Vargas Calderón, J. M., \& González Barrera, J. V. (2016). Uso del código QR para el seguimiento de la información del eslabón de producción de palma africana en el departamento del Casanare municipio de Villanueva mediante una prueba piloto en la finca Malybu. [Tesis de Grado].

\subsection{17. El código QR en el aprendizaje de Geografía e Historia}
\textbf{Resumen:} El problema que detonó este proyecto en el Instituto Santa Ana (Córdoba, Argentina) fue simple: un alto índice de alumnos desaprobados en Historia y Geografía de 6º año, lo que les impedía graduarse a tiempo. La solución que plantearon fue crear un Plan de Intervención para los profesores. La idea central es enseñar las dos materias de forma transversal usando el Código QR como herramienta tecnológica. El plan contempla cinco jornadas de capacitación con expertos. Los profesores reciben formación en: 1) El uso del QR para generar contenidos multimedia; 2) Cómo articular las materias y usar imágenes/textos de forma más significativa; y 3) Teorías de inteligencias múltiples y emocionales para entender cómo aprenden los adolescentes. El objetivo final es que los docentes dejen de dar clases tradicionales, aprovechen los recursos tecnológicos (proyectores, tablets) y logren que el alumno construya un aprendizaje significativo, reduciendo así la desaprobación en esas materias.

\textbf{Reflexión:} La universidad necesita dejar lo cuadrado. Al mezclar el QR con plataformas que los estudiantes sí usan, el aprendizaje se vuelve real y útil, desarrollando habilidades blandas vitales.

\textbf{Referencia:} Pérez, I. S. (2023). El código QR en el aprendizaje de Geografía e Historia. [Tesis de Licenciatura].

\subsection{18. Guía virtual interactiva en Android a través de códigos QR en el Museo de la Escuela Fiscal Isidro Ayora del Ecuador}
\textbf{Resumen:} El punto de partida de esta investigación fue el Museo de la Escuela Isidro Ayora en Latacunga, Ecuador, que estaba "olvidado" y era visitado de manera irregular porque exhibía sus artículos de forma "tradicional, monótona y poco llamativa". Para revivir este espacio y difundir su patrimonio, se propuso desarrollar una guía virtual interactiva en Android usando Códigos QR. La idea central es simple: se adhiere un Código QR a cada artículo exhibido. El visitante solo tiene que escanearlo con cualquier celular Android para acceder de inmediato a la información completa del objeto (descripción, fotos, multimedia, etc.). Para construir esto, el equipo usó la metodología de desarrollo ágil Mobile-D (ideal para smartphones por su rapidez y bajo costo). La implementación busca que el museo pase de ser un centro pasivo a uno "interactivo y amigable", atrayendo más visitas y logrando mayor reconocimiento.

\textbf{Reflexión:} Este proyecto prueba que la tecnología no es cara. El Código QR es la solución perfecta para convertir cualquier exposición monótona en algo interactivo y llamativo.

\textbf{Referencia:} Viscaino Naranjo, F., Rodríguez Bárcenas, G., Rubio Peñaherrera, J. B., Gualuiza, J., \& Carrillo, J. (2016). Guía virtual interactiva en Android a través de códigos QR en el Museo de la Escuela Fiscal Isidro Ayora del Ecuador. Ciencias de la Información, 47(3), 9–17.

\subsection{19. SD-EQR: Una nueva técnica para utilizar códigos QR™ en criptografía}
\textbf{Resumen:} El punto es que en la comunicación de hoy, cifrar mensajes es totalmente vital porque nadie quiere que un hacker o un espía interprete información sensible, como cuando un gerente manda una instrucción al banco. Para resolver este problema de seguridad, el autor propone una técnica llamada SD-EQR que combina la encriptación con la capacidad de almacenamiento del Código QR. El proceso es sofisticado: primero, se encripta el mensaje usando una "llave simétrica" que se genera a partir de una contraseña, y luego esa primera encriptación se vuelve a cifrar con otros métodos (como invertir la cadena y hacer operaciones XOR). El resultado final es un mensaje con varios niveles de seguridad que se convierte en uno o varios Códigos QR. La genialidad de esto es que el QR le añade una capa de seguridad extra, permitiendo enviar datos encriptados de forma rápida. Esta técnica sirve para todo: desde comunicaciones gubernamentales hasta para que una persona guarde su pasaporte o sus documentos importantes de forma perfectamente segura.

\textbf{Reflexión:} El Código QR funciona si le pones una encriptación de varios niveles, se vuelve un sistema de seguridad.

\textbf{Referencia:} Dey, S. (2012). Sd-eqr: Una nueva técnica para usar códigos QR en criptografía. arXiv preprint arXiv:1205.4829.

\subsection{20. Uso de los códigos Quick Response (QR) en instituciones de información}
\textbf{Resumen:} El uso del Código QR en instituciones de información (bibliotecas, museos, archivos, etc.) es todavía "muy, muy bajo", a pesar de que el mundo ya es digital y la gente joven (nativos digitales) usa el celular para todo. La investigación argumenta que estas instituciones desconocen las enormes ventajas del QR para ahorrar tiempo y recursos, tanto para el personal como para los usuarios. Los objetivos de este estudio fueron identificar el potencial del QR y cómo integrarlo a la Alfabetización Informacional (ALFIN). El QR es una herramienta genial porque minimiza el tiempo al dar acceso inmediato a información (sitios web, documentos, videos) sin tener que escribir nada. Esto permite a las bibliotecas ofrecer contenido de alto valor, sustituir las fichas de papel por un simple código y agilizar procesos internos. La gran conclusión es que las instituciones deben dejar su mentalidad "tradicional" y crear programas de ALFIN enfocados en el móvil para que los usuarios y especialistas puedan aprovechar a fondo esta tecnología.

\textbf{Reflexión:} Las bibliotecas, los archivos y los museos están con un problema grave de mentalidad. No tiene sentido que, en pleno 2025, con la gente pegada al celular, sigan con procesos de papel o marbetes antiguos.

\textbf{Referencia:} Paz Enrique, L. E. (2017). Uso de los códigos Quick Response (QR) en instituciones de información. Revista Publicando, 4(12 (1)), 3–15.

\section{Checklist de reproducibilidad}

\begin{itemize}
  \item Datos: resúmenes de artículos, porcentajes calculados.
  \item Código: scripts de análisis en code/.
  \item Entorno: LaTeX, Python.
  \item Procedimiento: síntesis manual.
  \item Resultados: tablas en build/.
\end{itemize}