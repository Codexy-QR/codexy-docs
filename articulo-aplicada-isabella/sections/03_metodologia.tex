\section{Metodología}

\subsection{Diseño del estudio}

Este estudio emplea una metodología de revisión sistemática de literatura aplicada, sintetizando 20 artículos seleccionados por su relevancia en códigos QR y gestión de inventarios. Se siguió un proceso de análisis cualitativo y cuantitativo, identificando patrones, tecnologías y resultados.

\subsection{Selección y análisis de artículos}

Los 20 artículos fueron seleccionados de bases de datos académicas, cubriendo periodos de 2008 a 2024. Se categorizaron por sector: inventarios generales (7), salud (3), agricultura (2), educación (3), otros (5). Para cada artículo, se extrajeron resúmenes, reflexiones, referencias y métricas de efectividad.

\subsection{Comparación de tecnologías QR}

Se realizó una comparación multi-criterio de implementaciones QR en diferentes contextos. La \cref{fig:comparacion_metodologias} ilustra la evaluación de efectividad por sector.

\begin{figure}[H]
    \centering
    \includegraphics[width=0.3\textwidth]{graphics/comparacion_metodologias.pdf}
    \caption{Comparación de efectividad de QR por sector}
    \label{fig:comparacion_metodologias}
\end{figure}

\subsection{Datos e instrumentos}

Se recopilaron porcentajes de efectividad, beneficios reportados y reflexiones de los artículos. Se utilizaron tablas para cuantificar resultados, como reducción de pérdidas y mejora en eficiencia.

Los datos abarcan 12 años de artículos, proporcionando una visión longitudinal.

\subsection{Procedimiento y validez}

El procedimiento incluyó: (1) lectura y extracción de datos, (2) categorización temática, (3) síntesis cuantitativa, (4) integración de reflexiones. La validez se asegura mediante referencias cruzadas y análisis objetivo de métricas.

Esta metodología garantiza una síntesis reproducible y comprehensiva de las experiencias con QR en inventarios.