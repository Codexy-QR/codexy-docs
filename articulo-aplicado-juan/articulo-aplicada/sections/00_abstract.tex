En un mundo que está creciendo en tecnología, debe comenzar a actualizarse, CODEXY busca modernizar la manera en cómo se hace inventario, buscando minimizar la utilización de papel, siendo más práctico y mucho más fácil para cualquier persona, con solo una escanear un código QR ese ítem ya está en inventario, desde la comodidad desde nuestro mismo teléfono, codexy busca generar una manera de que el inventario sea más rápido, eficientemente y práctico, tenido su parte administrativa, tenido en cuenta esto el aplicativo debe manejar gran cantidad de datos, manejar una seguridad y orientarse a un modelo el cual pueda ser eficiente para sus necesidades, no obstante se debe orientar a que el aplicativo maneje la big data de manera adecuada, haciendo un buen testing buscando modernizar con la ia, herramientas que nos harán más fácil el adecuado funcionamiento del aplicativo.
Este estudio explora la evolución del desarrollo de software, poniendo frente a frente a los monolitos y los microservicios, pero con un giro moderno: la integración de datos en tiempo real e Inteligencia Artificial generativa.

Dado el enorme volumen de datos actual y el uso de nubes híbridas, analizamos la importancia de trabajar con eventos (EDA) y agilidad DevOps. La clave está en encontrar el equilibrio entre simplicidad y flexibilidad; por eso investigamos prácticas nuevas como la 'Federación de Microservicios Monolítica' (MCMF), sumando seguridad Zero Trust y la velocidad del Edge Computing.

También integramos herramientas de IA (como GPT-4) para agilizar tareas críticas, como la creación de tests y la revisión del código. Los resultados confirman que unir estas tecnologías hace que el desarrollo sea mucho más eficiente. Finalmente, se ofrece una guía práctica para la elección arquitectónica y la automatización inteligente, discutiendo las implicaciones futuras para sistemas resilientes.
