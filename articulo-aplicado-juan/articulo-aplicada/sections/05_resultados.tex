\section{Resultados}
. La Automatización Inteligente del Ciclo de Desarrollo
La integración de la Inteligencia Artificial Generativa (IA Gen) representa la culminación de la automatización en el ciclo de vida del desarrollo de software. Si bien los microservicios y DevOps impulsan la agilidad en el despliegue, la IA Gen está transformando la calidad del producto y la eficiencia del equipo al delegar tareas cognitivamente intensivas a modelos de lenguaje avanzados.
Este enfoque se alinea con las tendencias de MLOps \cite{mlops_trends_2024}, que buscan integrar modelos de machine learning de manera ágil y continua en la producción, asegurando que la IA actúe como un socio dinámico y resiliente dentro del pipeline de CI/CD (Integración Continua/Entrega Continua).
V.2. IA Gen en Pruebas Automatizadas
El área de pruebas (Testing) es una de las que más se beneficia de la IA Gen, ya que aborda uno de los desafíos más costosos en el desarrollo: la creación y mantenimiento de casos de prueba y oráculos.
Un estudio comparativo reciente demostró que los modelos de lenguaje a gran escala (LLM, como GPT-4) pueden superar a las herramientas de automatización tradicionales (como Selenium) en la generación dinámica de casos de prueba y oráculos \cite{genai_testing_comparative_2024}.
•   Generación de Oráculos: La IA es capaz de predecir el resultado esperado de una prueba (el oráculo) basándose en la especificación o documentación, eliminando errores humanos en la definición del resultado correcto.
•   Aumento de la Cobertura: Al generar casos de prueba variados y a menudo inesperados, la IA Gen aumenta significativamente la cobertura de pruebas en comparación con los métodos manuales o basados en plantillas fijas.
Esta automatización reduce la latencia en el pipeline de desarrollo y permite a los equipos de DevOps alcanzar una velocidad de entrega superior sin sacrificar la calidad, un factor crítico en arquitecturas distribuidas donde los errores pueden propagarse rápidamente.
V.3. Optimización de Código y Detección de Defectos
Más allá del testing funcional, la IA Gen se aplica en el análisis estático y dinámico del código fuente, un proceso crucial para la prevención de errores y la mitigación de vulnerabilidades de seguridad.
Un enfoque utiliza Redes Neuronales Artificiales (ANN), optimizadas mediante técnicas específicas, para el escaneo de código fuente y la detección de defectos ocultos \cite{analysis_genai_source_scanning_2025}. El modelo es capaz de:
•   Identificar Patrones de Error: La ANN analiza patrones sintácticos y semánticos asociados con defectos de programación o vulnerabilidades de seguridad que podrían pasar desapercibidos en revisiones humanas o herramientas de análisis simples.
•   Mejorar Métricas de Calidad: Al sugerir refactorizaciones basadas en la detección de patrones ineficientes o la aplicación de técnicas de optimización (como la clasificación de errores en regresiones), la IA contribuye a mejorar métricas objetivas como la latencia y la utilización de recursos.
De esta manera, la IA Generativa se consolida como un mecanismo inteligente para elevar la calidad intrínseca de los artefactos de software, complementando las estrategias de seguridad como Zero Trust \cite{zero_trust_multicloud_2024} y garantizando sistemas más resilientes.

\begin{figure}[h]
\centering
\includegraphics[width=0.9\columnwidth]{graphics/performace_of_models_in_automated.jpg} % Ancho ajustado
\caption{Rendimiento de modelos en pruebas automatizadas}
\end{figure}