\section{Metodología}

La migración hacia arquitecturas orientadas a eventos (EDA) surge como respuesta al crecimiento exponencial de los datos y la necesidad de sistemas altamente desacoplados. En microservicios, donde las interacciones síncronas generan latencia y dependencias innecesarias, EDA emerge como la solución fundamental para la integración de datos en tiempo real. Un evento se define como un cambio de estado significativo en el sistema, lo que permite a los servicios reaccionar de manera asíncrona e inmediata.

La implementación de EDA en entornos de nube mejora la escalabilidad y la resiliencia, garantizando que los fallos en un servicio no paralicen toda la cadena de procesamiento de datos. Esta arquitectura no solo facilita la comunicación entre microservicios, sino que también es el motor que impulsa la analítica avanzada, el machine learning y la toma de decisiones automatizada.

La eficiencia de la EDA depende directamente de las herramientas de streaming y almacenamiento de datos. El análisis de tecnologías actuales identifica un stack dominante para el procesamiento de grandes volúmenes de datos en tiempo real:

Apache Kafka: Actúa como el message broker distribuido de alta capacidad, manejando el transporte de eventos con garantía de durabilidad y throughput.

Apache Flink: Se especializa en el procesamiento de streaming con estado (stateful), permitiendo transformaciones complejas y detección de patrones (CEP).

Redis: Capa de caching en memoria y almacenamiento de baja latencia.

La evolución de la arquitectura también se extiende al Edge Computing. Los ecosistemas de Big Data en ambientes híbridos y la proliferación de dispositivos IoT requieren procesamiento cada vez más cercano a la fuente de datos, mitigando latencia y ancho de banda.

Finalmente, esta necesidad de acceso rápido ha impulsado tendencias en bases de datos NoSQL y NewSQL que ofrecen mayor flexibilidad y escalabilidad horizontal para entornos distribuidos.
