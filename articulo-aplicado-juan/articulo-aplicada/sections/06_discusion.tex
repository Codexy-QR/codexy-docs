\section{Discusión de Resultados y Amenazas a la Validez}
La evaluación de las tecnologías presentadas —desde el uso de Consul e Istio \cite{service_discovery_networks_2024} hasta la optimización del rendimiento en Edge Computing \cite{event_driven_edge_2024}— demuestra mejoras significativas en la eficiencia y la reducción de errores operativos. No obstante, la implementación de estas soluciones no está exenta de amenazas a la validez que deben ser mitigadas:
•   Complejidad Inherente a la Distribución: La principal limitación sigue siendo la dificultad para gestionar la red y el estado en sistemas distribuidos. La latencia es una variable crítica que puede ser añadida por soluciones de seguridad (como los proxies de Service Mesh) o por la propia coordinación de eventos.
•   Curva de Aprendizaje y Costo: La adopción de tecnologías como Kafka, Flink o Istio exige una alta curva de aprendizaje y personal especializado, lo cual impacta el costo operativo y la eficiencia del equipo a corto plazo.
La mitigación de estas amenazas se aborda mediante una rigurosa observabilidad \cite{observability_microservices_2025}, que proporciona métricas detalladas para el diagnóstico, y una guía de implementación basada en la experiencia práctica \cite{microservice_to_monolith_2024}.