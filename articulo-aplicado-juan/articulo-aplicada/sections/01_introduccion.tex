\section{Introducción}
Durante la última década, el desarrollo de software ha dado un giro radical, empujado por la necesidad de agilidad y de procesar datos masivos al instante.

Codexy busca agilizar estos procesos en cuestión de hacer inventarios, ya que estos poseen una gran cantidad de datos lo cuales deben procesarse y manejarse de una manera segura ya que todo esto lo hace en tiempo real, al hacer esto, es propenso a tener errores, por ende su arquitectura, su seguridad, y su agilidad tanto en código como en desarrolladores debe de ser impecable.

Si bien los microservicios se consolidaron como el estándar para escalar y facilitar DevOps, esta transición complicó bastante la operación, sobre todo en la seguridad y la gestión de nubes híbridas. Quedó claro que las herramientas tradicionales ya no alcanzan.

Por eso, ahora ganan terreno soluciones como las arquitecturas orientadas a eventos (EDA) y la seguridad Zero Trust. A esto se suma la IA Generativa, que está cambiando las reglas del juego en el control de calidad (QA), automatizando pruebas y detectando fallos con una precisión que los métodos antiguos no tienen. En este artículo analizamos a fondo estas tendencias, revisando tanto sus soluciones como sus riesgos.
