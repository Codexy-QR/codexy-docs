\section{Conclusiones }
Conclusiones Centrales: La Convergencia de los Tres Pilares
Este estudio de investigación aplicada confirma que la evolución del desarrollo de software está marcada por una convergencia estratégica de tres pilares tecnológicos: la arquitectura híbrida, el procesamiento en tiempo real y la automatización inteligente con IA Gen.
1.  Hacia la Arquitectura Híbrida y Pragmática: La revisión concluye que la adopción de microservicios ya no es un fin en sí mismo, sino una herramienta que debe equilibrarse con la complejidad operativa. El surgimiento de modelos como la Federación de Microservicios Monolítica (MCMF) \cite{mcmf_2025} y la justificación de la "reversión a monolitos" \cite{microservices_transformation_2025} demuestran un pragmatismo arquitectónico. El éxito reside en la elección de la abstracción adecuada al dominio, tal como sugieren las revisiones sobre el diseño arquitectónico \cite{revisiting_abstractions_2025}.
2.  Optimización por Eventos y Datos en Vivo: La Arquitectura Orientada a Eventos (EDA), habilitada por herramientas de streaming (Kafka, Flink) y optimizada mediante Edge Computing, se establece como la base para la agilidad DevOps \cite{eda_realtime_2024}. El procesamiento de datos en tiempo real es una condición sine qua non para la toma de decisiones instantánea y la escalabilidad en entornos de IoT y multi-nube.
3.  Calidad Acelerada por IA y Seguridad Proactiva: La IA Generativa (GPT-4, ANN) ha demostrado ser crucial en la optimización del proceso, logrando mejoras significativas en métricas objetivas como la cobertura de pruebas y la reducción de errores de clasificación \cite{analysis_genai_source_scanning_2025, genai_testing_comparative_2024}. Paralelamente, la seguridad se transforma del modelo perimetral a un enfoque Zero Trust Security (ZTS) \cite{zero_trust_multicloud_2024}, que garantiza la integridad del sistema distribuido a través de la verificación continua.

Lecciones Aprendidas y la Guía Práctica
Las lecciones derivadas de esta revisión se sintetizan en una guía práctica reproducible que enfatiza la necesidad de un enfoque holístico, donde la tecnología debe ser complementada por la madurez organizacional y la gestión de personas.
1.  Priorizar la Decisión Arquitectónica (Guía Práctica): Antes de migrar, se debe realizar un análisis costo-beneficio para determinar si el problema requiere la complejidad de un microservicio o si un modelo híbrido, como MCMF, puede ofrecer un equilibrio superior entre simplicidad y flexibilidad \cite{mcmf_2025, revisiting_abstractions_2025}.
2.  Integrar la Automatización Inteligente: La IA Gen debe ser un componente esencial del pipeline de CI/CD para la generación de artefactos (pruebas, código scaneo) y no solo una herramienta de soporte.
3.  El Factor Humano es Crítico: Finalmente, el éxito de cualquier transformación tecnológica (adopción de DevOps, ZTS o IA) depende del equipo. Investigaciones confirman el impacto positivo de la Inteligencia Emocional en el liderazgo de equipos de desarrollo \cite{emotional_intelligence_leadership_2023}, lo que se traduce en una mejor gestión de la complejidad, mayor colaboración y, en última instancia, en una mayor calidad y eficiencia del producto.
Las futuras líneas de investigación deben centrarse en el desarrollo de modelos de IA más adaptativos, que puedan optimizar las redes de microservicios dinámicamente, y en la creación de herramientas de ZTS más resilientes con una latencia mínima.