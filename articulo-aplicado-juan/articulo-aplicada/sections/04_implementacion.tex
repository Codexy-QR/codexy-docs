\section{Implementación}

Zero Trust Security (ZTS): El Fin del Perímetro
La adopción de microservicios y el despliegue en ambientes multi-nube e híbridos han disuelto el concepto tradicional de "perímetro de red" seguro. En este contexto, un atacante que logra penetrar la red se mueve lateralmente sin restricción (movimiento lateral).
El modelo Zero Trust Security (ZTS) surge como el nuevo estándar de seguridad \cite{zero_trust_multicloud_2024}. Su principio fundamental es: "Nunca confíes, siempre verifica". Bajo ZTS, se asume que cada usuario, dispositivo y servicio que intenta acceder a un recurso es potencialmente hostil, independientemente de su ubicación dentro o fuera de la red corporativa.
La implementación de ZTS en arquitecturas de microservicios se centra en tres pilares clave \cite{zero_trust_multicloud_2024, access_control_models_2025}:
1.  Verificación Continua: Autenticación estricta y re-autorización en cada solicitud, utilizando múltiples factores y políticas de acceso dinámicas.
2.  Acceso con Mínimos Privilegios: Solo se otorga el acceso necesario para completar una tarea específica, limitando el posible impacto de una brecha.
3.  Microsegmentación: Se crean zonas de seguridad pequeñas y aisladas, asegurando que la comunicación entre dos microservicios solo sea posible si está explícitamente permitida.
IV.2. Gestión de Redes y Descubrimiento de Servicios
La microsegmentación requiere una gestión sofisticada de la comunicación entre los servicios. Herramientas de descubrimiento de servicios y Service Mesh son esenciales para implementar políticas de ZTS y gestionar el tráfico entre nubes \cite{service_discovery_networks_2024}.

% IMAGEN ZTS
\begin{figure}[h]
\centering
\includegraphics[width=0.9\columnwidth]{graphics/zts.jpg}
\caption{Modelo Zero Trust Security}
\end{figure}

•   Mecanismos de Descubrimiento: En un entorno dinámico, los servicios deben poder encontrarse automáticamente. Se comparan soluciones como el tradicional DNS con mecanismos avanzados como Consul e Istio. Istio, como malla de servicios (Service Mesh), es particularmente relevante, ya que inyecta proxies en la red que manejan el cifrado (mTLS), la autenticación y la aplicación de políticas de tráfico, actuando como el punto de aplicación de las políticas ZTS en la comunicación entre microservicios \cite{service_discovery_networks_2024}.
Esta infraestructura de red avanzada, combinada con el análisis de Big Data y las capacidades de IA \cite{bigdata_security_ia_2025}, permite establecer sistemas de seguridad adaptativos que pueden responder a amenazas en tiempo real.
IV.3. Observabilidad como Requisito de Resiliencia
Un sistema distribuido no solo debe ser seguro, sino también comprensible. La Observabilidad es la capacidad de inferir el estado interno de un sistema basándose en sus salidas externas (métricas, logs, trazas) \cite{observability_microservices_2025}. A diferencia del monitoreo tradicional (que mide fallos conocidos), la Observabilidad permite a los equipos diagnosticar fallos desconocidos y problemas de rendimiento.
En el contexto de los microservicios, la Observabilidad es crucial por varias razones \cite{observability_microservices_2025}:
•   Diagnóstico de Latencia: Permite rastrear la trayectoria de una solicitud a través de múltiples servicios (Distributed Tracing), identificando cuellos de botella y latencia introducida por el procesamiento de eventos o las llamadas a red.
•   Validación de ZTS: Los logs y trazas detalladas son fundamentales para auditar las políticas de acceso y verificar que las reglas de ZTS se están aplicando correctamente.
•   Gestión de Recursos: La Observabilidad impacta directamente en la eficiencia, ya que las métricas de utilización de recursos permiten optimizar el despliegue y reducir costos.
La convergencia de ZTS y Observabilidad asegura un ecosistema de microservicios que no solo está fortificado contra amenazas externas e internas, sino que también puede mantener un alto nivel de calidad de servicio y eficiencia operativa.