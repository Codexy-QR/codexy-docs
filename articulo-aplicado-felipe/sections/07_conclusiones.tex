\section{Conclusiones}
Al final del día, desarrollar Codexy nos enseñó que modernizar un inventario va mucho más allá de simplemente quitar el papel y poner una tablet. Después de revisar la arquitectura, tirar código y ver cómo funciona todo en la realidad, llegamos a tres conclusiones que son oro para cualquier proyecto de software hoy en día.

Primero, confirmamos que la arquitectura importa, y mucho. Decidirnos por .NET 9 con Clean Architecture en lugar de lanzarnos de cabeza a una red compleja de microservicios fue un acierto total. Como vimos en la teoría, los microservicios son muy potentes, pero para el tamaño de este proyecto, un monolito modular bien hecho nos dio la estabilidad que necesitábamos sin todo el caos operativo. Aprendimos que no siempre ``lo último que salió'' es lo mejor para todo; el contexto es el que manda.

Segundo, comprobamos que la agilidad y la calidad pueden ir de la mano. Usar metodologías ágiles nos permitió movernos rápido y adaptar el software a lo que los operativos de la bodega realmente necesitaban. Además, meter principios SOLID y pruebas automáticas con Selenium desde el arranque no fue una pérdida de tiempo, sino una inversión. Gracias a eso, hoy tenemos un sistema fuerte que no se rompe cada vez que queremos agregar algo nuevo.

Por último, y quizás lo más importante: la tecnología no arregla problemas de cultura. Codexy es una herramienta potente que centraliza datos y evita errores, pero su éxito depende 100\% de que las personas la adopten. La transformación digital es, en el fondo, una transformación humana. El software deja la mesa servida para el futuro —pensando ya en Inteligencia Artificial y prácticas sostenibles—, pero el cambio verdadero ocurre cuando el equipo entiende que escanear ese código QR es parte vital de su trabajo y no una tarea más.

En resumen, Codexy demuestra que, con la arquitectura correcta y pensando siempre en el usuario, es posible transformar el caos de un inventario manual en una ventaja real para la empresa.