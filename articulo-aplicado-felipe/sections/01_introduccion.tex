\section{Introducción}
Estamos en una época donde la tecnología vuela. Desde que aparecieron las primeras computadoras, las empresas han tenido que correr para no quedarse atrás, pero lo que ha pasado en los últimos años, especialmente después de la pandemia de COVID-19, fue un cambio total. Como bien dice Páez Navarro (2023), ya no se trata solo de tener una página web o usar el correo electrónico; hoy en día, las empresas que no se transforman digitalmente corren el riesgo de desaparecer o de perder su capacidad para competir en el mercado.

Y esto no pasa solo en otros lados, aquí en Latinoamérica también se siente. Por ejemplo, en Uruguay ya se está viendo que la mayoría de los ejecutivos están metiendo automatización para agilizar procesos y bajar costos, tal como mencionan Rodríguez et al. (2024). La cosa está clara: hay que buscar la eficiencia operativa sí o sí para sobrevivir en una economía donde los márgenes de ganancia son cada vez más apretados.

Aun con toda esta revolución, es increíble que muchas instituciones grandes sigan sufriendo por algo tan básico como el desorden en sus inventarios. Todavía es común ver organizaciones que dependen de conteos a mano, hojas de papel o archivos de Excel que no cuadran para saber qué tienen en bodega. El problema de esto no es solo que el proceso sea lento, sino que puede salir caro. Según Parra Ángel y Fuentes Rojas (2022) en su análisis sobre el control de materiales, la falta de un sistema claro genera pérdidas de dinero directas, atrasos en los proyectos y gastos extra que se podrían evitar simplemente sabiendo qué material entra y cuál sale.

Justo para atacar ese problema nace Codexy.

Este proyecto surge para resolver esa falta de claridad y los típicos errores de seguir usando ``lápiz y papel'' en lugares con estructuras complejas. El problema que enfrentamos no era solo de código, sino de operación: no había trazabilidad real del historial del inventario, la gente de campo no tenía herramientas ágiles para reportar y los encargados perdían tiempo valioso tratando de unir datos que ya estaban viejos.

La idea de este artículo es mostrarles el desarrollo y la arquitectura de Codexy. A diferencia de las soluciones de siempre, este sistema apuesta por la modernización mediante el uso de códigos QR para la captura rápida de datos y una arquitectura de software robusta basada en .NET 9 y tecnologías móviles híbridas. No queríamos simplemente ``digitalizar'' un proceso manual, sino reestructurar la forma en que la institución gestiona sus activos, garantizando seguridad, rapidez y, sobre todo, datos confiables para la toma de decisiones. A lo largo de este documento, se explicará cómo se ha unido metodologías ágiles y patrones de diseño modernos para construir una herramienta que no solo funciona bien en lo técnico, sino que realmente le sirva a la gente que la usa todos los días.