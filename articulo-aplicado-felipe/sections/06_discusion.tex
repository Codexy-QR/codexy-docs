\section{Discusión}
Hasta acá hemos hablado maravillas de .NET 9, de que la arquitectura quedó limpia y de lo rápido que vuela SignalR. Técnicamente, Codexy es una máquina bien aceitada. Pero si somos brutalmente honestos, la tecnología por sí sola no hace milagros. Después de ver los resultados y compararlos con lo que leímos, nos dimos cuenta de que el éxito o el fracaso de esto no depende de si el código compila bonito, sino de cosas humanas y estratégicas que son más complicadas.

\subsection{El Factor Humano: La Barrera Invisible}
Aquí es donde la cosa se pone difícil. Páez Navarro (2023) da en el clavo cuando dice que la transformación digital es, antes que nada, cultural. Puedes tener el software más rápido del mundo, pero si la gente no sabe usarlo o, peor, no quiere usarlo, la inversión se va a la basura.

Con Codexy nos estrellamos con una realidad dura: la calidad de los datos depende de la gente. Parra y Fuentes lo advierten claro: el punto débil siempre es el registro humano. Si al operativo en la bodega le da pereza sacar el celular para escanear el QR y prefiere anotarlo en un papel para pasarlo ``luego'', el sistema pierde toda la gracia del tiempo real.

Pero hay algo más profundo que la pereza: el miedo. Como se vio en el estudio de Uruguay, meter automatización asusta porque la gente cree que va a perder su empleo. Si los encargados ven a Codexy como un ``policía digital'', lo van a sabotear sin que te des cuenta. Por eso, la discusión no es solo sobre qué código usar, sino sobre cómo enseñamos a usarlo. Hay que vender la tecnología como una herramienta que te quita el trabajo aburrido de contar a mano, no como algo que viene a quitarte el puesto.

\subsection{Análisis Costo-Beneficio: Lo que Cuesta el Desorden}
Hablemos de dinero, porque al final las empresas invierten para ver retorno. A veces cuesta justificar gastar en un desarrollo como Codexy en vez de seguir con Excel. Pero cuando miras casos como el de ``Realidad Colombia SAS'' que analiza Parra Angel (2022), ves que el desorden sale carísimo: materiales que se pierden, obras atrasadas y compras dobles de emergencia.

Codexy ataca el ``Costo de la Ignorancia''. ¿Cuánto le cuesta a la empresa no saber que tiene 500 sillas guardadas en el sótano y comprar otras 500 por error? Montar este sistema puede parecer caro al principio, pero el retorno de inversión se dispara cuando dejas de tener ``inventario fantasma''. Al usar el QR, bajamos el error humano casi a cero y convertimos la bodega en una fuente de información real, no en un agujero negro de gastos.

\subsection{De la Digitalización a la Inteligencia (Hacia la Industria 4.0)}
Ahora, miremos hacia adelante. Codexy hoy resuelve el ``¿Qué tengo?'', pero su verdadero poder está en los datos que está guardando para mañana. Estamos preparando el terreno para la Inteligencia Artificial.

Peñalver e Isea-Argüelles (2024) explican que la IA es el motor del futuro, permitiendo predecir mantenimientos y optimizar todo. Pero el secreto es este: la IA no sirve sin datos históricos buenos. Hoy, Codexy está recolectando datos de forma masiva y ordenada. Mañana, con toda esa historia en la base de datos, podríamos meter un módulo de IA que nos diga: ``Oye, según lo que has gastado este año, te vas a quedar sin papel en 3 semanas''. Estamos construyendo la base para una ``institución inteligente''. No puedes saltar al futuro sin haber ordenado la casa primero, que es justo lo que hace Codexy.

\subsection{Sostenibilidad y Ética: Tecnología Verde}
Por último, hay algo que los desarrolladores a veces olvidamos por estar corriendo: el impacto ambiental. Al eliminar los reportes en papel y mejorar los procesos, Codexy entra en la onda de Green IT.

Como propone Reina Guaña, usar tecnología para gestionar procesos no es solo para ahorrar plata, sino para reducir la huella ecológica. Al usar una arquitectura eficiente en .NET 9 (que gasta menos servidor y energía) y evitar que la gente tenga que viajar físicamente solo para verificar un inventario (porque ya lo ven en la Web), estamos ayudando a un modelo más sostenible. Puede parecer poco, pero en empresas grandes, ahorrar toneladas de papel y energía suma bastante. Es hacer tecnología con conciencia.